\documentclass{article}
\usepackage[utf8]{inputenc}
\usepackage[backend=biber, style=nature, citestyle=nature, maxnames=7, minnames=7, maxcitenames=2, mincitenames=1]{biblatex}
\usepackage[colorlinks=true,  urlcolor=blue]{hyperref}

\addbibresource{thesis.bib}
\title{Introduction}
\date{\vspace{-5ex}}

\begin{document}
\maketitle
\section{NFDS in \textit{I. elegans}}
Three different female phenotypes have been found in \textit{I. elegans}, most notably differing in colour scheme. The Androchrome phenotype (A) is blue, like the males of \textit{I. elegans}. This in combination with the narrower abdomen makes them phenotypically similar to males, and this phenotype is assumed to be a 'male mimic'. The other two phenotypes therefore can collectively be referred to as heterochrome. These phenotypes are the green Infuscans (I) and the orange Infuscans-Obsoleta (O). While the exact genetical processes generating these different phenotypes are unknown, crossing experiments have indicated that most likely a single autosomal locus with three alleles is responsible for the polymorphism\cite{Cordero1990}. The allele for the androchrome phenotype (p) is dominant over both other alleles, with the allele for infuscans (q) dominant over the allele for infuscans-obsoleta (r)\cite{Cordero1990}. 

\printbibliography
\end{document}
