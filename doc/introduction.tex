\documentclass{article}
\usepackage[utf8]{inputenc}
\usepackage[backend=biber, style=nature, citestyle=nature, maxnames=7, minnames=7, maxcitenames=2, mincitenames=1]{biblatex}
\usepackage[colorlinks=true,  urlcolor=blue]{hyperref}

\addbibresource{thesis.bib}
\title{Introduction}
\date{\vspace{-5ex}}

\begin{document}
\maketitle

\section{General -- Remove header later}
Selection as a driving force of evolution has been studied extensively since it was first described over 150 years ago \cite{Darwin1859, Wallace1858}. The main focus in this research has generally been on directional change over time. Much of the research in this field has dealt with both sexes adapting in response to each other, and especially effects such as Fisherian runaways \cite{Fisher1915}. It was however shown using a modeling approach that unidirectional adaptation is not the only possible outcome of sexual selection \cite{Gavrilets2002}, with the alternative possibility of multiple female phenotypes evolving. Two different 'responses' from males were predicted, either having a single phenotype which is forced to choose a female morph for mating, or evolving the same number of different phenotypes adapted to an individual female morph. Since then evidence for this type of sexual selection has been found in various species \cite{Hardling2006,Kagawa2016,Svensson2007}.

\section{Sexual Conflict}
\cite{Chapman2003}


\section{Bringing it down to \textit{I. elegans}}
Polymorphisms in one sex as a result of sexual conflict occurs in a variety of species \cite{Needed}. One clade in which these polymorphisms are particularly common are Odonata (Dragonflies and Damselflies). Males of this clade are often aggressive in their mating behaviour, and optimal mating rates for males are higher than they are for females. In addition, while many mating attempts fail, there is evidence that failed mating attempts also have negative effects on female fitness. As a result, avoiding detection by males is beneficial to the females. In many species different colour variations in the females have evolved, either multiple variations unique to females, or mimicking the male colour scheme. The latter option has the additional benefit of being less attractive due to potential confusion with males. In many of these species evidence has been found of negative frequency dependence, with the least common morph having a higher fitness.

\section{NFDS in \textit{I. elegans}}
In the blue-tailed damselfly (\textit{Ischnura elegans}) in particular, three different colour variations are found in females. No adaptation from males in response has been detected so far. The Androchrome phenotype (A) is blue with a blue tail-spot, like the males of \textit{I. elegans}. This in combination with a slightly narrower abdomen makes them phenotypically similar to males, and this phenotype is assumed to be a 'male mimic'. The other two phenotypes therefore can collectively be referred to as heterochrome. These phenotypes are the green Infuscans (I) and the orange-brown Infuscans-Obsoleta (O). In both of these morphs the blue tail-spot is present before maturity, but in mature females this is covered with melanin, leading to a brown patch. While the exact genetical processes generating these different phenotypes are unknown, crossing experiments have indicated that most likely a single autosomal locus with three alleles is responsible for the polymorphism\cite{Cordero1990}. The allele for the androchrome phenotype (p) is dominant over both other alleles, with the allele for infuscans (q) dominant over the allele for infuscans-obsoleta (r) \cite{Cordero1990}.

As there is also evidence that males learn to recognise females rather than having an innate preference, this leads to increased mating harassment for females whose phenotype is common in the population. As this harassment reduces the fertility of these females, it leads to negative frequency-dependent selection (NFDS), reducing the fitness of the most common phenotype and increasing that of the least common.
There is however strong evidence that beside negative-frequency dependence, male mimicry plays a big role in mating harassment as well, with androchrome females being harassed less. It has been found that a larger proportion of infuscans females is found mating than androchrome females, regardless of the morph frequencies \cite{Gosden2009}. In addition, the blue patch on the eighth abdominal segment which is present in males, androchrome females and immature heterochrome females has been experimentally shown to reduce mating harassment \cite{Willink2019}. This indicates that both negative frequency dependence and male mimicry play a role in sexual selection.

\section{Frequency Distribution}
In a study investigating populations throughout Europe, it was found that over 80\% of populations contained all three phenotypes, while not a single population was completely monomorphic \cite{Gosden2011}. While the absence of monomorphic populations may be explained by NFDS, the reasons for the prevalence of trimorphic populations are less obvious. In addition, while trimorphic populations occur all throughout Europe, there are definite patterns in the morph distribution, with androchrome frequency increasing with increasing latitudes \cite{Gosden2011}. Nevertheless there are also large differences between geographically close populations \cite{Gosden2011}. In addition, there is significant variation between subsequent years in the same populations \cite{Svensson2005}. This indicates the effect of drift should be relatively strong on these populations, and as such the presence of multiple morphs is a clear indicator of strong selective pressures.
Various other differences between the female morphs have been found to affect their fitness, potentially explaining the temporal and spatial variation in morph frequencies. The androchrome females have been found to have lower fecundity than heterochrome females, potentially due to their narrower abdomen \cite{Gosden2009}. Different morphs also have different behaviour to avoid mating harassment, with androchromes showing less resistance to mating attempts \cite{Gosden2009}. There are also differences between morphs relating to parasite infection, both in terms of the infection intensity and the effects of infection on fecundity \cite{Willink2017}.
While these effects have been described individually, the way they interact is poorly understood.

\printbibliography
\end{document}
