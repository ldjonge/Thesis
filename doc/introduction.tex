\documentclass{article}
\usepackage[utf8]{inputenc}
\usepackage[backend=biber, style=nature, citestyle=nature, maxnames=7, minnames=7, maxcitenames=2, mincitenames=1]{biblatex}
\usepackage[colorlinks=true,  urlcolor=blue]{hyperref}

\addbibresource{thesis.bib}
\title{Introduction}
\date{\vspace{-5ex}}

\begin{document}
\maketitle
\section{NFDS in \textit{I. elegans}}
Three different female phenotypes have been found in \textit{I. elegans}, most notably differing in colour scheme. The Androchrome phenotype (A) is blue, like the males of \textit{I. elegans}. This in combination with the narrower abdomen makes them phenotypically similar to males, and this phenotype is assumed to be a 'male mimic'. The other two phenotypes therefore can collectively be referred to as heterochrome. These phenotypes are the green Infuscans (I) and the orange Infuscans-Obsoleta (O). While the exact genetical processes generating these different phenotypes are unknown, crossing experiments have indicated that most likely a single autosomal locus with three alleles is responsible for the polymorphism\cite{Cordero1990}. The allele for the androchrome phenotype (p) is dominant over both other alleles, with the allele for infuscans (q) dominant over the allele for infuscans-obsoleta (r)\cite{Cordero1990}.

The maintenance of the different phenotypes is attributed to sexual selection as a result of mating harassment. Males will repeatedly attempt to mate with females, with a relatively low success rate. There is however strong evidence for negative effects on the female as a result of a failed mating attempt. As there is also evidence that males learn to recognise females rather than having an innate preference, this leads to increased mating harassment for females whose phenotype is common in the population. As this harassment reduces the fertility of these females, it leads to negative frequency-dependent selection (NFDS), reducing the fitness of the most common phenotype and increasing that of the least common.

\section{Frequency Distribution}
In a study investigating populations throughout Europe, it was found that over 80\% of populations contained all three phenotypes, while not a single population was completely monomorphic \cite{Gosden2011}. This would indicate a high degree of balancing selection, which could be explained by NFDS related to mating harassment. It should however be noted that even in the case of NFDS, this would not necessarily lead to trimorphic populations, as the same effects could be found equally in populations where only two phenotypes are present. In addition, it was found that while the frequency distributions differ between populations, there is a clear pattern with more 

\printbibliography
\end{document}
