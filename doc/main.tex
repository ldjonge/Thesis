\documentclass{article}
\usepackage[utf8]{inputenc}

\usepackage[backend=biber, style=nature, citestyle=nature, maxnames=7, minnames=7, maxcitenames=2, mincitenames=1]{biblatex}
\addbibresource{thesis.bib}

\title{MSc Thesis}
\author{Lorenzo de Jonge}
\date{April 2020}

\begin{document}

\maketitle

\section{Abstract}

\section{Introduction}

\subsection{General Sexual Conflict}
Selection as a driving force of evolution has been studied extensively since it was first described over 150 years ago \cite{Darwin1858, Darwin1859}. The main focus in this research has generally been on directional change over time. Much of the research in this field has dealt with both sexes adapting in response to each other, and especially effects such as Fisherian runaways \cite{Fisher1915}. It was however shown using a modeling approach that unidirectional adaptation is not the only possible outcome of sexual selection \cite{Gavrilets2002}, with the alternative possibility of multiple female phenotypes evolving. Two different 'responses' from males were predicted, either having a single phenotype which is forced to choose a female morph for mating, or evolving the same number of different phenotypes adapted to an individual female morph. Since then evidence for this type of sexual selection has been found in various species \cite{Hardling2006,Kagawa2016,Svensson2007}.

\cite{Chapman2003}


\subsection{Bringing it down to \textit{I. elegans}}
Polymorphisms in one sex as a result of sexual conflict occurs in a variety of species \cite{Needed}. One clade in which these polymorphisms are particularly common are Odonata (Dragonflies and Damselflies). Males of this clade are often aggressive in their mating behaviour, and optimal mating rates for males are higher than they are for females. In addition, while many mating attempts fail, there is evidence that failed mating attempts also have negative effects on female fitness. As a result, avoiding detection by males is beneficial to the females. In many species different colour variations in the females have evolved, either multiple variations unique to females, or mimicking the male colour scheme. The latter option has the additional benefit of being less attractive due to potential confusion with males. In many of these species evidence has been found of negative frequency dependence, with the least common morph having a higher fitness.

\subsection{NFDS in \textit{I. elegans}}
In the blue-tailed damselfly (\textit{Ischnura elegans}) in particular, three different colour variations are found in females. No adaptation from males in response has been detected so far. The Androchrome phenotype (A) is blue with a blue tail-spot, like the males of \textit{I. elegans}. This in combination with a slightly narrower abdomen makes them phenotypically similar to males, and this phenotype is assumed to be a 'male mimic'. The other two phenotypes therefore can collectively be referred to as heterochrome. These phenotypes are the green Infuscans (I) and the orange-brown Infuscans-Obsoleta (O). In both of these morphs the blue tail-spot is present before maturity, but in mature females this is covered with melanin, leading to a brown patch. While the exact genetic processes generating these different phenotypes are unknown, crossing experiments have indicated that most likely a single autosomal locus with three alleles is responsible for the polymorphism\cite{Cordero1990}. The allele for the androchrome phenotype (p) is dominant over both other alleles, with the allele for infuscans (q) dominant over the allele for infuscans-obsoleta (r) \cite{Cordero1990}. There is evidence in the closely related \textit{I. senegalensis}, which has only one heterochrome morph, that the splicing of the \textit{doublesex (dsx)} gene is responsible for the female limited colour polymorphisms \cite{Takahashi2019}. This could also be the case in \textit{I. elegans}?

As there is also evidence that males learn to recognise females rather than having an innate preference, this leads to increased mating harassment for females whose phenotype is common in the population. As this harassment reduces the fertility of these females, it leads to negative frequency-dependent selection (NFDS), reducing the fitness of the most common phenotype and increasing that of the least common.
There is however strong evidence that beside negative-frequency dependence, male mimicry plays a big role in mating harassment as well, with androchrome females being harassed less. It has been found that a larger proportion of infuscans females is found mating than androchrome females, regardless of the morph frequencies \cite{Gosden2009}. In addition, the blue patch on the eighth abdominal segment which is present in males, androchrome females and immature heterochrome females has been experimentally shown to reduce mating harassment \cite{Willink2019}. This indicates that both negative frequency dependence and male mimicry play a role in sexual selection.

\subsection{Frequency Distribution}
In a study investigating populations throughout Europe, it was found that over 80\% of populations contained all three phenotypes, while not a single population was completely monomorphic \cite{Gosden2011}. While the absence of monomorphic populations may be explained by NFDS, the reasons for the prevalence of trimorphic populations are less obvious. In addition, while trimorphic populations occur all throughout Europe, there are definite patterns in the morph distribution, with androchrome frequency increasing with increasing latitudes \cite{Gosden2011}. Nevertheless there are also large differences between geographically close populations \cite{Gosden2011}. In addition, there is significant variation between subsequent years in the same populations \cite{Svensson2005}. This indicates the effect of drift should be relatively strong on these populations, and as such the presence of multiple morphs is a clear indicator of strong selective pressures.
Various other differences between the female morphs have been found to affect their fitness, potentially explaining the temporal and spatial variation in morph frequencies. The androchrome females have been found to have lower fecundity than heterochrome females, potentially due to their narrower abdomen \cite{Gosden2009}. Recent research has however shown that androchrome females develop to maturity faster, and were less sensitive to lower temperatures \cite{Svensson2019}.
Different morphs also have different behaviour to avoid mating harassment, with androchrome females showing less resistance to mating attempts \cite{Gosden2009}. There are also differences between morphs relating to parasite infection, both in terms of the infection intensity and the effects of infection on fecundity \cite{Willink2017}.
While these effects have been described individually, the way they interact is poorly understood.

\subsection{Migration}
An important factor to consider in this model is the dispersal of individuals. Not only is migration the only way a new population can be formed, it also provides possibilities for gene transfer between populations. Dispersal of Odonata between ponds is thought to be largely a result of random movement, rather than intentional migration \cite{conrad2002, Moore1954}.
There is evidence that female \textit{I. elegans} disperse further than males, however no difference between the female morphs has been found \cite{conrad2002}.
Pond characteristics appear to have limited effects on survival, however they do have a significant effect on dispersal, with evidence that urban ponds lead to reduced dispersal rate and distance \cite{Gall2017}, possibly due to less sun availability and more shadow.

\subsection{Differential Extinction}
A differential extinction risk based on the number of morphs present in a population could explain why as of yet no monomorphic populations have been found. It has been shown in the past that species with polymorphisms are less likely to populate a new area \cite{bessa2003},
If populations with fewer morphs present are at a higher risk of going extinct, they may be extinct before they are found. May use \cite{Ovaskainen2010} to calculate extinction risk


\subsection{General Introduction Stuff}

Female colour morphs in many species

Well recorded data of 3 female morphs in \textit{Ischnura elegans}, all 3 morphs present in over 80\% of populations investigated. None of the other populations were monomorphic \cite{Gosden2011}. The frequency distribution of the morphs was however highly variable between the populations.
Mostly univoltine in northern populations\cite{Svensson2007}, but bivoltine has occurred occasionally.

With population sizes relatively small \textbf{SOURCE NEEDED}, it could be expected that many populations would reach fixation due to drift. This however has not been observed, and out of 100+ populations in Europe all have at least 2 morphs present, with most containing all 3 morphs \cite{Gosden2011}. This implies strong balancing selection.

3 in this species \textit{Ischnura elegans}, 1 andromorph 2 others. Andromorph also shares other physical traits with males, leaner etc. Lower fecundity.
No signs of male adaptation on a genetic level, however plastic learning. Other species such as diving beetles have shown genetic adaptation to change in other sex, maybe here as well? \cite{Karlsson2013}.

Evidence of effect of these genes on male development \cite{Abbott2005, Abbott2007}


Experimental work is being done to find out more about this species, theoretical background with modeling. Individual based models are nice but limited population size, population genetic models may be better? Try both for optimal results. Basic pop-gen model has been done \cite{LeRouzic2015}, expand to include more things.

\section{Methods}
\subsection{System}
The model was based on \textit{Ischnura elegans}, a small damselfly species with a natural range from the UK in the west to Japan in the east, and from Scandinavia in the north to India in the south \cite{IUCNisch}.
in which males are monomorphic, and females have three colour morphs, one androchrome and two heterochrome.


As shown in a recent study \cite{Svensson2019}, females belonging to different colour morphs mature at different speeds, affected by temperature differences. Androchromes were shown to mature faster, and were affected less by low temperatures \cite{Svensson2019}. In natural populations, maturing faster will increase the probability of living to maturity, due to reduced time for predation.

\subsection{General Model Stuff}
An individual-based model was created to evaluate the evolution of colour morph frequencies over time. For this model in each generation a population of male and female individuals was created. The genotype was simulated for a single locus with three different alleles, with genotype having no phenotypical effect on males. In females the genotype affects the colour morph, where the three alleles have a clear dominance hierarchy.  The three alleles are referred to as \textit{p}, \textit{q} and \textit{r}. Any females with one or multiple \textit{p} alleles would display the fully dominant Androchrome (A) phenotype, \textit{qq} and \textit{qr} females the Infuscans (I) phenotype, and \textit{rr} females display the fully recessive Obsoleta (O) phenotype. These morphs most have visual differences in \textit{I. elegans}, which are used for male learnt mate recognition. The A females resemble males and are as such considered male mimics. Beside the colour differences, different morphs were also considered to have differences in fecundity and survivability in the pre-mating phase.

\subsection{Learning / Male Response}
In this model males were assumed to have no inherent preference for any female morph. Instead, they learn to recognise 'favourable' mates by interacting with them and attempt to mate. In case the female accepts the mating happens and the male will increase his preference for the morph of the female interacted with. In this case both the male and female are also removed from the mating pool for the next 2 iterations, as copulation takes time during which neither individual is available for mating with others. If the female rejects mating, the male will have a reduced preference for her morph. As Androchrome females are considered male mimics, there is a probability for males to be mistaken for androchrome females and the other way around. Any interaction with another male will have the same effect on androchrome preference as a failed attempt at mating with an androchrome female would. The probability for a male to mistake another male for an androchrome female is fixed, and the preference for males is thus directly related to the preference for A females.

\subsection{Female response}
Females can respond to mating attempts either by accepting or rejecting the mating. In case she accepts, they mate and the female will be fertilised. Nonetheless, her fecundity will go down as a result of the time and effort spent, and the harassment from the male. In case of rejection, no further interaction occurs, but the female's fecundity will be lowered as a result of mating harassment. Females are assumed to start out with a relatively low probability of accepting a mating. As they grow older, until a female mates she become more likely to accept a mating, to prevent dying without having fertilised eggs. This behaviour is based on the observation that females belonging to a rarer morph in real populations appear to be more promiscuous \cite{Citation Needed}. This can be assumed to be a result of reduced mating harassment due to the lower probability of being recognised as a potential mate by males. Once mating has happened, the probability of accepting a mating attempt is reduced to the original starting value, as the value of additional matings can be assumed to be relatively low. The probability should however not be set to 0, as there is clear evidence of females mating with multiple females \cite{Cooper1996}.

\subsection{Reproduction}
When a mating occurs, there is a transfer of genetic material from the male to the female. Field evidence suggests that \textit{I. elegans} females store the sperm, and will lay eggs only once even in case they mate with multiple males. Due to males removing the sperm from previous matings, if a female has mated with multiple males around 80\% of the sperm comes from the final mating \cite{Cooper1996, Cordoba2003}. Therefore the offspring will for 80\% be fathered by the final male, with the other 20\% divided equally between any previous males. All fertilised females lay a number of eggs equal to their fecundity. There is an equal division of eggs between all possible genotypes based on alleles from both parents, with half of the offspring male and the other half female.

The number of eggs surviving to become mature individuals in the next generation is determined based on the number of eggs laid and the carrying capacity of the population, with random variation to represent different environmental circumstances in different years. These eggs are then randomly selected from the pool of available eggs, with no selection based on the colour morph genotype. Due to the random selection the exactly equal distributions of sex and phenotype in the population of eggs are randomised in the mating population, introducing drift into the model.

\subsection{Interactions}
For each generation the mating season is split up in a fixed number of cycles. In each cycle all males have the opportunity to interact with a female. A male has a subset of up to 10 individuals (reduced in smaller populations) with which interactions are possible, as in reality a male will not have a choice between all individuals in a population. Depending on the size of the sample and the preferences for the individuals a male may not interact with any others, with a lower chance of interactions in a smaller sample and in samples with 'unpreferred' morphs. If a male does interact, an individual is chosen from the sample based on the preference for each individual. The male attempts to mate with the chosen individual, who can either accept or reject. If the mating is rejected, the male's preference for the chosen morph is reduced, and the target's fertility is reduced as a result of mating harassment. In females this is represented by a direct effect on the number of eggs laid, in males this is represented by a reduction in future mating success. If the mating is successful, the female is fertilised, and the male's preference for the chosen morph is increased. Nonetheless, there is a reduction in the female's fecundity as a result of the usually less than gentle behaviour associated with mating attempts. In addition, both male and female are removed from the mating pool for the next few cycles, as mating takes more time than failed mating attempts.

\subsection{Migration}

To investigate the effects of migration multiple populations were simulated at the same time, between which migration could occur. Individuals were allowed to migrate to a different population according to specified migration rates either before or after the mating phase. In case of migration before the mating phase, males and females could migrate to interact with individuals in the new population and potentially mate there. In case of migration after the mating phase, females who were fertilised in the population of origin could lay their eggs in a different population, leading to the flow of genes of both herself and any males she mated with before migrating.

These simulations could be used to study the possible introduction of a new colour morph into existing populations, as well as the potential starting of a new population by including a 'population' of size 0 into which individuals could migrate. In all cases the number of migrations was recorded both before and after the mating phase.

\subsection{Environment}

To study interactions between environmental selection pressures and sexual selection pressures, environmental effects can be modelled in various ways. Fecundity can be affected by the environment, as could survival as an adult, or survival from the larval stage to the adult stage. For this model the survival from larval to adult stage was used to simulated environmental effects, consistent with evidence from the model species \cite{Svensson2019}. For these simulations the number of individuals reaching maturity was calculated as described above, with the probability of each individual larva reaching maturity differing per morph. The mating population was then generated from the egg population using weighted random sampling. As no data on males was present and males are monomorphic, the male maturation chance was taken as a baseline.

\subsection{Recording}

For each generation in a simulation various results are recorded for each population. The population size is recorded both to evaluate potential effects of drift and to evaluate the population fitness, where shrinking populations can be assumed to have a lower overall fitness. The frequency of each morph at the start of a generation is recorded, to evaluate the evolution of morph frequency without the effects of potential deaths during the mating season. Mate preference averaged over all males at the end of the season is also recorded, to evaluate the effects of varying morph frequencies and other factors on the preference of males. Female fecundity is recorded as a direct proxy of the negative effects of mating harassment. As females have a baseline fecundity that is only affected by the harassment from males, the strength of the harassment effects can be directly deduced.

\section{Results}

\subsection{Model Validation}
Trimorphic populations can be maintained without any genetic preference, or any learning before the first mating attempts. Simply having males and females interact, with males adjusting their preference based on the success of previous attempts, will lead to a (weakly) sigmoid, but clearly positive, relationship between morph frequency and the preference of males for that morph.

The fecundity of each morph is negatively correlated with the frequency at which the morph is present, as well as with the male preference for this morph. Female fecundity is reduced as a result of male harassment, which is a direct consequence of the way the model is set up.

\subsection{Female fecundity}
As female fecundity was affected both by the baseline fecundity of the morph and the harassment suffered from males, fecundity was recorded per morph. As expected with all morphs present at equal frequencies androchrome females had a lower fecundity than either gynochrome morph. Both gynochrome morphs, when present at the same frequency, had similar fecundity levels. As expected if one gynochrome morph was present at a higher frequency than the other, this morph had a lower fecundity as a result of increased mating harassment. Notably when populations were near equilibrium frequencies, all three morphs had similar fecundity. This indicates that when no abiotic selection occurs, negative frequency dependent selection will lead to the morph with a lower baseline fitness to have the frequency reduced until the point where the harassment on the other morphs is high enough to counteract this so all morphs are equally fit.

The female fecundity averaged over the three morphs was also studied in relation to the morph frequencies. No significant relationships were found between any of the morph frequencies and the female fecundity, although populations with a morph frequency of 0 or 1 for any morph did appear to have a lower fecundity on average. This indicates that polymorphic populations have a higher overall fitness compared to monomorphic populations.

\subsection{Population Size}
It was found that when the same carrying capacity is assumed, polymorphisms increase the overall fecundity levels of females. Trimorphic populations were found to be significantly larger than dimorphic populations of any morph combination ($p<10^{-10}$), which were in turn significantly larger than any monomorphic population($p<10^{-10}$). Due to the reduced baseline fecundity of the androchrome morph, dimorphic populations with A present were significantly smaller than gynochrome dimorphic populations ($p<10^{-10}$), and monomorphic A populations were significantly smaller than monomorphic gynochrome populations ($p<10^{-10}$), between which no significant difference was found.

\subsection{Migration}
The introduction of a single individual morph into an existing population where this morph is not present can be enough for this morph to be maintained in the population in the long term. Due to the way NFDS interacts with phenotype rather than genotype directly, phenotypes coded by dominant alleles are more likely to be maintained directly as they are more likely to be expressed in subsequent generations and as such the effects of selection are stronger. This is the case even if a phenotype has a reduced baseline fitness, as the effects of NFDS can overcome this.

Nevertheless, even phenotypes coded by recessive alleles have a probability of being introduced successfully into populations, and relatively low migration rates are enough for monomorphic populations to become dimorphic or even trimorphic if these populations are connected with existing trimorphic populations.

\subsubsection{New populations}
Relatively high migration rates are needed for new populations to form, as it appears the migration of a single individual, even if this is a fertilised female, is rarely if ever enough to form a new population. In addition, the simulations showed that any newly formed population that persisted more than a few generations was polymorphic within 3 generations. As the migration rates required for a new population to form are much higher than those required to introduce a new morph into a population, newly formed populations are highly likely to be polymorphic. It is also possible that due to the reduced fitness of females in monomorphic populations any newly formed monomorphic populations have a much lower probability of being maintained, but this needs to be tested further.

\subsection{Abiotic factors}

Adjusting the maturation rate of different morphs has the expected effect on morph frequency, where morphs with a higher maturation rate will have a higher frequency. When the differences in rates are big, the less successful morph will normally go extinct within a relatively short time period. If the differences are smaller however, populations will go to an equilibrium frequency distribution which can still contain all three morphs, with each frequency adjusted based on the maturation rate.

\section{Discussion}

\subsection{Answering Original Questions}
Differential migration rates have not been studied. With no empirical evidence this seems justifiable.
It has been found that polymorphic populations have significantly larger equilibrium population sizes compared to monomorphic populations, with additional morphs increasing female fitness and consequently population size. However, at equal equilibrium population sizes, extinction risk does not seem to differ significantly between populations of different morph compositions, with the notable exception of monomorphic androchrome populations. These populations appear to have a significantly increased risk of extinction at equal equilibrium population size compared to any other populations.

Environmental factors as introduced in this model lead to directional selection, as one or two morphs is by design favoured over the other regardless of morph frequencies. These simulations however have shown that the effects of directional selection do not necessarily exclude the balancing selection caused in this case by sexual selection. Instead the two selection pressures balance out to lead to equilibrium frequencies which may differ from a situation in which no environmental selection is present. Therefore, it is a possible explanation for the differences in the frequencies found in populations throughout Europe. Accounting for the faster growth rate and higher resistance to lower temperatures during maturation that androchromes possess \cite{Svensson2019} increases the equilbrium frequency of the androchrome morph, consistent with the higher androchrome frequencies found in colder areas.

\subsection{Unexpected Additional Findings}
While morph frequencies differ across Europe, many populations have a distinct difference in morph frequency between the Infuscans and Obsoleta morphs \cite{Gosden2011}. Additionally, while morph frequencies may differ significantly per season \cite{Gosden2011, LeRouzic2015}, these frequency shifts are rarely unidirectionally \cite{Gosden2011}, suggesting frequencies either shift around an equilibrium or adjust to environmental effects, rather than moving to an equilibrium. These simulations however have shown that two morphs which are identical in all ways other than learnt mate recognition have identical equilibrium frequencies, suggesting an as of yet unknown difference must exist between the two heterochrome morphs.

\subsection{Limitations}
This model does not in any way account for interspecies competition. Carrying capacities of populations are assumed to be constant, implying that newly formed populations can grow almost logistically as there is no competition limiting the growth. In a real population many times other damselfly species would be present which may compete for the same food sources. Additionally it has been observed repeatedly that damselfly males attempt to mate with individuals of different species. This would have an effect both on the males and females of \textit{I. elegans} being harassed by males from other species, but also on males which may mistakenly harass individuals of another species and as such fail to find a conspecific female.

% \nocite{*} %

\printbibliography

\end{document}
