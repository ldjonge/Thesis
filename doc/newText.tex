\section{Aims}
\\begin{itemize}
  \item Simulate the effects of morph fitness differing due to abiotic factors on the morph frequencies, aiming to confirm conclusions drawn from field research, e.g. \cite{Willink2019b}
  \item Simulate the effects of differing morph frequencies on the extinction probabilities of populations
  \item Simulate potential founder effects, seeing if new populations could truly be formed from a single female and still lead to trimorphic populations
\end{itemize}

\section{Background}
In previous studies combining field data with simulations, frequency distributions that optimise population fitness were obtained \cite{Svensson2005}. These values however do not coincide with the real frequency distribution found in the same populations the field data was obtained from, and the morph frequency vary widely throughout Europe, although a geographical pattern is clearly present \cite{Whoever wrote that}. While it has been suggested that the phenotypical differences between the three morphs are more extensive than just colour, the hypothesis that this can explain the geographical pattern has not been tested properly. This study will aim to do exactly that, combining existing field data with a newly built model incorporating the effects of both environmental factors and mating harassment on female fitness.
