\documentclass[10pt,letterpaper]{article}

\usepackage{changepage}

\usepackage[right]{lineno}

\usepackage{microtype}
\DisableLigatures[f]{encoding = *, family = * }

\newlength\savedwidth

\newcommand\thickcline[1]{%
  \noalign{\global\savedwidth\arrayrulewidth\global\arrayrulewidth 2pt}%
  \cline{#1}%
  \noalign{\vskip\arrayrulewidth}%
  \noalign{\global\arrayrulewidth\savedwidth}%
}

\newcommand\thickhline{\noalign{\global\savedwidth\arrayrulewidth\global\arrayrulewidth 2pt}%
\hline
\noalign{\global\arrayrulewidth\savedwidth}}

\usepackage[utf8x]{inputenc}

\usepackage[top=0.85in,left=2.75in,footskip=0.75in]{geometry}

\usepackage{cite}
\bibliographystyle{plos2015}

\usepackage{nameref,hyperref}

\raggedright
\setlength{\parindent}{0.5cm}
\textwidth 5.25in
\textheight 8.75in

\usepackage[aboveskip=1pt,labelfont=bf,labelsep=period,justification=raggedright,singlelinecheck=off]{caption}
\renewcommand{\figurename}{Fig}

\makeatletter
\renewcommand{\@biblabel}[1]{\quad#1.}
\makeatother

\usepackage{lastpage,fancyhdr,graphicx}
\usepackage{epstopdf}
%\pagestyle{myheadings}
\pagestyle{fancy}
\fancyhf{}
%\setlength{\headheight}{27.023pt}
%\lhead{\includegraphics[width=2.0in]{PLOS-submission.eps}}
\rfoot{\thepage/\pageref{LastPage}}
\renewcommand{\headrulewidth}{0pt}
\renewcommand{\footrule}{\hrule height 2pt \vspace{2mm}}
\fancyheadoffset[L]{2.25in}
\fancyfootoffset[L]{2.25in}
\lfoot{\today}

\title{Methods}
\date{\vspace{-5ex}}

\begin{document}

\maketitle

\section{System}
The model was based on \textit{Ischnura elegans}, a small damselfly species with a natural range from the UK in the west to Japan in the east, and from Scandinavia in the north to India in the south \cite{IUCNisch}.
in which males are monomorphic, and females have three colour morphs, one androchrome and two heterochrome.

\section{Model}
An individual-based model was created to evaluate the evolution of colour morph frequencies over time. For this model in each generation a population of male and female individuals was created. The genotype was simulated for a single locus with three different alleles, with phenotypic effects present in females, where the three alleles have a clear dominance hierarchy. Genotype had no phenotypical effect on males. The three alleles are referred to as \textit{p}, \textit{q} and \textit{r}. Any females with one or multiple \textit{p} alleles would display the Androchrome (A) phenotype, \textit{qq} and \textit{qr} females the Infuscans (I) phenotype, and \textit{rr} females display the Obsoleta (O) phenotype. These morphs most notably have visual differences, which are used for male learned mate recognition. The A females resemble males and are as such considered male mimics. Beside the colour differences, different morphs were also considered to have differences in fecundity and survivability in the pre-mating phase.

\subsection{Learning}
Males have no inherent preference for a specific female morph. Instead, they learn to recognise 'favourable' females by interacting with them. Males will interact with females and attempt to mate. In case the female accepts the mating happens and the male will increase his preference for the morph of the female interacted with. If the female rejects mating, the male will have a reduced preference for her morph. As Androchrome females are considered male mimics, there is a probability for males to be mistaken for androchrome females and the other way around. Any interaction with another male will have the same effect on androchrome preference as a failed attempt at mating with an androchrome female would. The probability for a male to mistake another male for an androchrome female is fixed, and the preference for males is thus a fraction of the preference for A females.

\subsection{Female response}
Females can respond to mating attempts either by accepting or rejecting the mating. In case she accepts, they mate and the female will be fertilised. Nonetheless, her fecundity will go down as a result of the time and effort spent, and the harassment from the male. In case of rejection, no further interaction occurs, but the female's fecundity will be lowered as a result of mating harassment. Females are assumed to start out with a relatively low probability of accepting a mating. As they grow older, until a female mates she become more likely to accept a mating, to prevent dying without having fertilised eggs. Once mating has happened, the probability of accepting a mating attempt is reduced to the original starting value. There is clear evidence of females mating with multiple females \cite{Cooper1996} and as such the probability should not be set to 0, however it can be expected to be reduced after a female has mated.

\subsection{Reproduction}
When a mating occurs, there is a transfer of genetic material from the male to the female. Field evidence suggests that \textit{I. elegans} females store the sperm, and will lay eggs only once even in case they mate with multiple males. Due to males removing the sperm from previous matings, if a female has mated with multiple males around 80\% of the sperm comes from the final mating \cite{Cooper1996, Cordoba2003}. Therefore the offspring will for 80\% be fathered by the final male, with the other 20\% divided equally between any previous males. All fertilised females lay a number of eggs equal to their fecundity. There is an equal division of eggs between all possible genotypes based on alleles from both parents, with half of the offspring male and the other half female.

The number of eggs surviving to become mature individuals in the next generation is determined based on the number of eggs laid and the carrying capacity of the population, with random variation to represent different environmental circumstances in different years. These eggs are then randomly selected from the pool of available eggs, with no selection based on the colour morph genotype.

As shown in a recent study \cite{Svensson2019}, females belonging to different colour morphs mature at different speeds, affected by temperature differences. Androchromes were shown to mature faster, and were affected less by low temperatures \cite{Svensson2019}. In natural populations, maturing faster will increase the probability of living to maturity, due to reduced time for predation. These differences were represented in the model by having the individuals surviving to maturity sampled with weighting based on the colour morph. As no data on males was present, the male chance was taken as baseline, with only one morph allowed to differ per simulation(?).

\subsection{Interactions}
For each generation the mating season is split up in a fixed number of cycles. In each cycle all males have the opportunity to interact with a female. A male has a subset of up to 10 individuals (reduced in smaller populations) with which interactions are possible, as in reality a male will not choose between all individuals in a population. Depending on the size of the sample and the preferences for the individuals a male may not interact with any others, with a lower chance of interactions in a smaller sample and in samples with 'unpreferred' morphs. If a male does interact, an individual is chosen from the sample based on the preference for each individual. The male attempts to mate with the chosen individual, who can accept or reject. If the mating is rejected, the male's preferences for the chosen morph is reduced, and the target's fertility is reduced as a result of mating harassment. In females this is simulated as a direct effect on the number of eggs laid, in males this is simulated as a reduction in future mating success. If the mating is successful, the female is fertilised, and the male's preference for the chosen morph is increased. Nonetheless, there is a reduction in the female's fecundity as a result of the usually less than gentle behaviour associated with mating attempts. In addition, both male and female are removed from the mating pool for the next few cycles, as mating takes more time than failed mating attempts.

\subsection{Recording}
For each generation in a simulation various results are recorded. The frequency of each morph at the start of a generation is recorded, to evaluate the evolution of morph frequency without the effects of potential deaths during the mating season. Mate preference averaged over all males at the end of the season is also recorded, to evaluate the effects of varying morph frequencies and other factors on the preference of males. Female fecundity is recorded as a direct proxy of the negative effects of mating harassment. As females have a baseline fecundity that is only affected by the harassment from males, the strength of the harassment effects can be directly deduced.

\section{Limitations}
This model does not in any way account for interspecies competition. Carrying capacities of populations are assumed to be constant, implying that newly formed populations can grow almost logistically as there is no competition limiting the growth. In a real population many times other damselfly species would be present which may compete for the same food sources. Additionally it has been observed repeatedly that damselfly males attempt to mate with individuals of different species. This would have an effect both on the males and females of \textit{I. elegans} being harassed by males from other species, but also on males which may mistakenly harass individuals of another species and as such fail to find a conspecific female.

\bibliography{thesis.bib}

\end{document}
